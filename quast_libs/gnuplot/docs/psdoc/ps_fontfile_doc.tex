\documentclass[a4paper,10pt]{article}
\usepackage[T1]{fontenc}
\usepackage{booktabs}
\usepackage{longtable}
\usepackage{graphicx}
\usepackage{array}
\usepackage{calc}
%\usepackage[width=18cm,height=24cm]{geometry}
\addtolength{\textwidth}{40mm}
\addtolength{\oddsidemargin}{-15mm}

\let\orgcaption\caption
\newlength\tmplength
\newcommand{\captionabove}{%
  \setlength{\tmplength}{\abovecaptionskip}%
  \setlength{\abovecaptionskip}{\belowcaptionskip}%
  \setlength{\belowcaptionskip}{\tmplength}%
  \orgcaption}

\font\cmmi=cmmi10
\font\cmr=cmr10
\font\cmti=cmti10
\font\cmtt=cmtt10
\font\cmu=cmu10
\font\cmsy=cmsy10
\font\cmex=cmex10
\font\cmss=cmss10
\font\cmff=cmff10
\font\cmtex=cmtex10
\font\lasy=lasy10

\newcount\rownum
\newcounter{dezcnt}

\def\dez#1#2#3\empty{\setcounter{dezcnt}{64*#1+8*#2+#3}\thedezcnt}

\newcommand*\row[1]{%
  \global\let\zus=\empty
  \ifnum#1=000
    \gdef\zus{241}%
  \else\ifnum#1=001
    \gdef\zus{242}%
  \else\ifnum#1=002
    \gdef\zus{243}%
  \else\ifnum#1=003
    \gdef\zus{244}%
  \else\ifnum#1=004
    \gdef\zus{245}%
  \else\ifnum#1=005
    \gdef\zus{246}%
  \else\ifnum#1=006
    \gdef\zus{247}%
  \else\ifnum#1=007
    \gdef\zus{250}%
  \else\ifnum#1=010
    \gdef\zus{251}%
  \else\ifnum#1=011
    \gdef\zus{252}%
  \else\ifnum#1=012
    \gdef\zus{255}%
  \else\ifnum#1=013
    \gdef\zus{256}%
  \else\ifnum#1=014
    \gdef\zus{257}%
  \else\ifnum#1=015
    \gdef\zus{260}%
  \else\ifnum#1=016
    \gdef\zus{261}%
  \else\ifnum#1=017
    \gdef\zus{262}%
  \else\ifnum#1=020
    \gdef\zus{263}%
  \else\ifnum#1=021
    \gdef\zus{264}%
  \else\ifnum#1=022
    \gdef\zus{265}%
  \else\ifnum#1=023
    \gdef\zus{266}%
  \else\ifnum#1=024
    \gdef\zus{267}%
  \else\ifnum#1=025
    \gdef\zus{270}%
  \else\ifnum#1=026
    \gdef\zus{271}%
  \else\ifnum#1=027
    \gdef\zus{272}%
  \else\ifnum#1=030
    \gdef\zus{273}%
  \else\ifnum#1=031
    \gdef\zus{274}%
  \else\ifnum#1=032
    \gdef\zus{275}%
  \else\ifnum#1=033
    \gdef\zus{276}%
  \else\ifnum#1=034
    \gdef\zus{277}%
  \else\ifnum#1=035
    \gdef\zus{300}%
  \else\ifnum#1=036
    \gdef\zus{301}%
  \else\ifnum#1=037
    \gdef\zus{302}%
  \else\ifnum#1=040
    \gdef\zus{303}% and \200
  \else\ifnum#1=177
    \gdef\zus{304}%
  \fi\fi\fi\fi\fi\fi\fi
  \fi\fi\fi\fi\fi\fi\fi
  \fi\fi\fi\fi\fi\fi\fi
  \fi\fi\fi\fi\fi\fi\fi
  \fi\fi\fi\fi\fi\fi
  \textbackslash #1%
  \ifx\zus\empty
  \else
    , \textbackslash\zus
  \fi & 
  {\cmr\char'#1} & 
  {\cmti\char'#1} & 
  {\cmtt\char'#1} & 
  {\cmmi\char'#1} &
  {\cmu\char'#1} &
  {\cmss\char'#1} &
  {\cmtex\char'#1} &
  {\cmff\char'#1} &
  {\cmsy\char'#1} &
  {\lasy\char'#1} &
  \ifodd\rownum \else\qquad\fi
  \global\advance\rownum 1
  {\raisebox{1.7ex}[0mm][1ex]{\cmex\char'#1}} &
  \textbackslash #1%
  \ifx\zus\empty
  \else
    , \textbackslash\zus
  \fi & 
  \expandafter\dez#1\empty
  \ifx\zus\empty
  \else
    , \expandafter\dez\zus\empty
  \fi \\
}

\begin{document}
\title{Using \TeX\ Fonts in the Gnuplot Postscript Terminal}
\author{Harald Harders, {\ttfamily h.harders@tu-bs.de}}
\date{2003-03-03}
\maketitle

\noindent
The Postscript terminal can embed Postscript Type\,1 fonts (with
extensions \verb|.pfa| and \verb|.pfb|) and TrueType fonts (extension
\verb|.ttf|)\footnote{If \texttt{.pfb} and \texttt{.ttf} fonts really
  can be embedded depends on your gnuplot installation: It needs to be
  able to handle pipes.} using the command
\begin{verbatim}
set terminal postscript fontfile '<filename>'
\end{verbatim}
The \verb|fontfile| option can be used multiple times.
See the sections \emph{set terminal postscript} and \emph{set
  fontpath} in the Gnuplot documentation for further description.

The embedded font can be used by 
\begin{verbatim}
set terminal postscript '<fontname>' <size>
\end{verbatim}
or in postscript enhanced terminal as following example:
\begin{verbatim}
set xlabel '{/CMMI10 x}'
\end{verbatim}

Among other things, the font embedding is useful for generating plots
to be included in \LaTeX\ documents. 
For normal text, the \emph{cm-super} Postscript Type\,1 fonts are a
good choice. 
They are available from CTAN servers, e.g.
\begin{verbatim}
ftp://ftp.dante.de/tex-archive/fonts/ps-type1/cm-super/
\end{verbatim}
The normal upright font with serifes is defined in
\verb|sfrm1000.pfb|, and the font name is \verb|SFRM1000|\footnote{If you
  have an old version of the cm-super font, prior 2001-10-14, the font
  name is in lowercase letters: \texttt{sfrm1000}. You should update
  to a new version.} (The \verb|1000| means that this font is designed
for 10\,pt).
Replace the \verb|rm| by \verb|it|, \verb|bx| or other combinations in
both the file name and the font name (here, in uppercase letters) in order 
to get other font shapes.
The encoding of these fonts is ordinary and thus is not described
here.
Table~\ref{tab:fontnames} shows some examples of fonts contained in
the cm-super font bundle.
%
\begin{table}
  \centering
  \captionabove{Some fonts in the cm-super font bundle (for a
    designsize of 10\,pt)}%
  \label{tab:fontnames}%
  \begin{tabular}{>{\ttfamily}l>{\ttfamily}ll}
    \toprule
    \multicolumn{1}{l}{File name}&
    \multicolumn{1}{l}{Full font name} &
    Example \\
    & \multicolumn{1}{l}{(all preceded by \texttt{Computer Modern})} & \\
    \midrule
    sfrm1000.pfb& Roman &
    {\rmfamily\upshape Example} \\
    sfbx1000.pfb& Bold Extended &
    {\rmfamily\bfseries\upshape Example} \\
    sfti1000.pfb& Italic &
    {\rmfamily\itshape Example} \\
    sfbi1000.pfb& Bold Extended Italic &
    {\rmfamily\bfseries\itshape Example} \\
    sfsl1000.pfb& Slanted &
    {\rmfamily\slshape Example} \\
    sfbl1000.pfb& Bold Extended Slanted &
    {\rmfamily\bfseries\slshape Example} \\
    sfcc1000.pfb& Caps and Small Caps &
    {\rmfamily\bfseries\scshape Example} \\
    \midrule
    sfss1000.pfb& Sans Serif &
    {\sffamily\upshape Example} \\
    sfsi1000.pfb& Sans Serif Slanted &
    {\sffamily\slshape Example} \\
    sfsx1000.pfb& Sans Serif Bold Extended &
    {\sffamily\bfseries\upshape Example} \\
    sfso1000.pfb& Sans Serif Bold Extended Slanted &
    {\sffamily\bfseries\slshape Example} \\
    \midrule
    sftt1000.pfb& Typewriter &
    {\ttfamily\upshape Example} \\
    sfit1000.pfb& Typewriter Italic &
    {\ttfamily\itshape Example} \\
    sfst1000.pfb& Typewriter Slanted &
    {\ttfamily\slshape Example} \\
    sftc1000.pfb& Typewriter Caps and Small Caps &
    {\ttfamily\scshape Example} \\
    \bottomrule
  \end{tabular}
\end{table}

For mathematics the Type\,1 versions of the Computer Modern fonts are
useful.
They should be installed in most \TeX\ implementations and are also
available from CTAN servers, e.g.
\begin{verbatim}
ftp://ftp.dante.de/tex-archive/fonts/cm/ps-type1/bluesky/pfb/
\end{verbatim}
Here, the font name is the base of the file name in uppercase letters,
e.g.\ the file \verb|cmmi10.pfb| contains the font \verb|CMMI10|.
Since the encoding of these fonts is strange, a table containing all
characters for some fonts follows.
The font \verb|CMEX10| contains large symbols for mathematics. They
overlap sometimes in the table. Since the baseline of the
\verb|CMEX10| font is at the top of the signs, Gnuplot defines a font
\verb|CMEX10-Baseline| with a different baseline if \verb|CMEX10| is
embedded (normally by using \verb|fontfile 'cmex10.pfb'|.
In contrast to the other fonts, \verb|CMEX10| is only available in the
design size 10\,pt.

You can access all characters of the fonts by typing their octal code.
To get a $\heartsuit$ symbol, you may type:
\begin{verbatim}
set label '{/CMSY10 \176}' at graph 0.5,0.5
\end{verbatim}
Since characters with an octal number below \textbackslash 040 can't
be displayed by some postscript interpreters, these characters are
repeated in the Computer Modern Fonts with a larger code.
Thus, you should use the larger number, where two octal numbers are
given (e.g.\ \textbackslash 000, \textbackslash 241).
For example, you better use
\begin{verbatim}
set xlabel '{/CMR10 \242}'
\end{verbatim}
than
\begin{verbatim}
set xlabel '{/CMR10 \001}'
\end{verbatim}
to get an upright uppercase Delta $\Delta$. 

\begin{longtable}{lllllllllllllr}
  \toprule
  Oct& \rotatebox{90}{CMR10}& \rotatebox{90}{CMTI10}&
  \rotatebox{90}{CMTT10}& \rotatebox{90}{CMMI10}&
  \rotatebox{90}{CMU10}& \rotatebox{90}{CMSS10}&
  \rotatebox{90}{CMTEX10}& \rotatebox{90}{CMFF10}&
  \rotatebox{90}{CMSY10}& \rotatebox{90}{LASY10}&
  \rotatebox{90}{CMEX10-Baseline}& Oct& Dec\\
  \midrule
  \endhead
  \bottomrule
  \endfoot
  \row{000}\row{001}\row{002}\row{003}\row{004}\row{005}\row{006}\row{007}
  \row{010}\row{011}\row{012}\row{013}\row{014}\row{015}\row{016}\row{017}
  \row{020}\row{021}\row{022}\row{023}\row{024}\row{025}\row{026}\row{027}
  \row{030}\row{031}\row{032}\row{033}\row{034}\row{035}\row{036}\row{037}
  \row{040}\row{041}\row{042}\row{043}\row{044}\row{045}\row{046}\row{047}
  \row{050}\row{051}\row{052}\row{053}\row{054}\row{055}\row{056}\row{057}
  \row{060}\row{061}\row{062}\row{063}\row{064}\row{065}\row{066}\row{067}
  \row{070}\row{071}\row{072}\row{073}\row{074}\row{075}\row{076}\row{077}
  %
  \row{100}\row{101}\row{102}\row{103}\row{104}\row{105}\row{106}\row{107}
  \row{110}\row{111}\row{112}\row{113}\row{114}\row{115}\row{116}\row{117}
  \row{120}\row{121}\row{122}\row{123}\row{124}\row{125}\row{126}\row{127}
  \row{130}\row{131}\row{132}\row{133}\row{134}\row{135}\row{136}\row{137}
  \row{140}\row{141}\row{142}\row{143}\row{144}\row{145}\row{146}\row{147}
  \row{150}\row{151}\row{152}\row{153}\row{154}\row{155}\row{156}\row{157}
  \row{160}\row{161}\row{162}\row{163}\row{164}\row{165}\row{166}\row{167}
  \row{170}\row{171}\row{172}\row{173}\row{174}\row{175}\row{176}\row{177}
  %
\end{longtable}

\end{document}
